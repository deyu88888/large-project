\chapter{Objectives and Stakeholders}

\section{Project Objectives}

The primary objective of this project is to create a cohesive digital ecosystem that strengthens university community engagement through improved society management and participation. This objective can be broken down into the following components:

\begin{itemize}
    \item \textbf{Enhancing student participation} in extracurricular activities by simplifying discovery and reducing barriers to society involvement
    \item \textbf{Enabling efficient management} of university societies through specialized tools for leadership roles
    \item \textbf{Creating meaningful connections} among students with similar interests who might otherwise not interact
    \item \textbf{Providing administrative oversight} for university staff to ensure societies operate within institutional guidelines
\end{itemize}

The realization of these objectives delivers value to the university community by enriching student experiences beyond academic pursuits, fostering skill development through organized activities, and strengthening institutional community bonds.

\section{Stakeholders}

\subsection{Stakeholder Analysis}

Our stakeholder analysis identified several groups affected by or influencing the implementation of this platform:

\begin{description}
    \item[Students] As the primary users and beneficiaries, students gain improved access to extracurricular activities that complement their academic experience. The platform allows them to discover societies aligned with their interests through AI-driven recommendations, easily join groups, track events, and connect with peers. Students' university experience is directly enhanced through increased participation opportunities.
    
    \item[Society Leaders] Presidents, vice presidents, and event managers benefit from specialized tools that simplify administrative tasks. They can manage membership requests, organize events, publish news, and communicate with members efficiently. This reduces their administrative burden and allows them to focus on creating valuable experiences for their members.
    
    \item[University Administrators] The administrative staff gains enhanced oversight of all student activities through a comprehensive dashboard. They can ensure compliance with university policies, approve new societies and events, and maintain quality standards. The platform streamlines their workflow by centralizing student organization management.
    
    \item[University as an Institution] The university benefits from increased student engagement and satisfaction, which can positively impact retention rates and institutional reputation. A vibrant student community with active societies contributes to a positive campus environment and enhances the university's appeal to prospective students.
    
    \item[Academic Staff] Faculty members indirectly benefit when students develop complementary skills through society participation. These extracurricular activities can reinforce classroom learning and provide practical application opportunities for academic concepts.
    
    \item[Potential Students] Prospective students considering enrollment decisions are influenced by the richness of campus life and available activities. A well-organized society ecosystem, visible through the platform, serves as a recruitment asset.
    
    \item[External Partners] Organizations that collaborate with university societies (such as sponsors, community groups, or industry partners) benefit from more organized interaction through the platform's structured communication channels.
    %should I remove this?
    \item[Alumni] Former students maintain connections to their university societies through the platform, potentially increasing alumni engagement and support.
\end{description}

The implementation of this platform creates different value propositions for each stakeholder group. For students, the impact is immediate and direct—they gain personalized access to activities that enhance their university experience. Society leaders transform their management capabilities through specialized tools that reduce administrative friction. University administrators receive unprecedented visibility into student activities, allowing for better resource allocation and policy enforcement.

\subsection{Key Stakeholders}

Based on our analysis, we identified the following key stakeholders whose interests and requirements significantly shape the project outcomes:

\begin{itemize}
    \item \textbf{Students} are the primary users whose engagement determines the platform's success. Their requirements for intuitive navigation, personalized recommendations, and social connectivity directly influence design decisions. The platform's value is primarily measured through student adoption and participation rates.
    
    \item \textbf{Society Leaders} are crucial for content generation and community building within the platform. Their ability to effectively manage membership, events, and communications directly impacts student experience. Their satisfaction with management tools is essential for sustainable platform growth.
    
    \item \textbf{University Administrators} hold significant influence over platform implementation and institutional adoption. Their oversight requirements regarding policy compliance, appropriate content, and resource allocation shape the platform's governance features. Administrative approval mechanisms are integrated throughout the system based on their operational needs.
\end{itemize}

These key stakeholders formed the core focus during requirements gathering and feature prioritization. Regular consultation with representatives from each group ensured that the platform addresses their specific needs while balancing overall project objectives.