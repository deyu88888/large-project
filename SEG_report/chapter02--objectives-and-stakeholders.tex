\chapter{Objectives and Stakeholders}

\section{Project Objectives}

Our project aims to create a digital hub that strengthens university community spirit through better society management and student participation. We've broken this down into several key goals:

\begin{itemize}
    \item \textbf{Boosting student involvement} in extracurricular activities by making societies easier to find and join
    \item \textbf{Streamlining society management} with tailored tools for student leaders
    \item \textbf{Fostering new friendships} between students with shared interests who might otherwise never meet
    \item \textbf{Supporting university oversight} so staff can ensure societies follow institutional guidelines
\end{itemize}

Achieving these objectives adds genuine value to university life by enriching students' time beyond lectures and seminars, helping them develop practical skills through organised activities, and building stronger community bonds across the institution.

\section{Stakeholders}

\subsection{Stakeholder Analysis}
Our analysis identified several groups who would be affected by or have influence over our platform:
\begin{description}
    \item[Students] As the main users and beneficiaries, students get better access to activities that complement their academic life. The platform helps them discover societies that match their interests, join groups without hassle, keep track of events, and connect with their peers. Their university experience improves through more opportunities to get involved.
    \item[Society Leaders] Committee members benefit from purpose-built tools that take the headache out of administrative tasks. They can handle  society management without much hassle. This frees them up to focus on creating worthwhile experiences for their members.
    
    \item[University Administrators] Admin staff gain better visibility of student activities through a comprehensive dashboard. They can monitor society compliance with university policies, approve new societies and events, and maintain quality standards. The platform makes their job easier by bringing society management together in one place.
    
    \item[The University] The institution itself benefits from higher student engagement and satisfaction, which can improve retention rates and enhance its reputation. A lively student community with active societies creates a positive campus atmosphere and makes the university more attractive to potential applicants.
    
    \item[Academic Staff] Lecturers and tutors benefit indirectly when students develop complementary skills through society activities. These extracurricular pursuits can reinforce classroom learning and provide practical applications for academic concepts.
    
    \item[Prospective Students] Sixth-formers and others considering where to study are influenced by the richness of campus life and available activities. A well-organised society ecosystem, visible through the platform, helps with recruitment.
    
    \item[External Partners] Organisations that work with university societies (such as sponsors, community groups or industry partners) benefit from more structured interaction through the platform's communication channels.
\end{description}
Our platform creates different benefits for each stakeholder group. For students, the impact is immediate—they gain personalised access to activities that enhance their university experience. Society leaders transform how they run things through tools that cut administrative faff. University administrators get unprecedented visibility into student activities, enabling better resource allocation and policy enforcement.

\subsection{Key Stakeholders}

Based on our analysis, we identified three key stakeholder groups whose needs and requirements significantly shaped our project:

\begin{itemize}
    \item \textbf{Students} are the primary users whose engagement determines whether the platform succeeds or fails. Their need for intuitive navigation, personalised recommendations and social connectivity directly influenced our design decisions. We measure the platform's value primarily through student adoption and participation.
    
    \item \textbf{Society Leaders} are vital for generating content and building community within the platform. Their ability to effectively manage membership, events and communications directly impacts the student experience. Their satisfaction with our management tools is essential for sustainable growth.
    
    \item \textbf{University Administrators} hold significant sway over platform implementation and institutional adoption. Their requirements around policy compliance, appropriate content and resource allocation shaped our governance features. We integrated administrative approval mechanisms throughout the system based on their operational needs.
\end{itemize}

These key stakeholders were our main focus during requirements gathering and feature prioritisation. Regular chats with representatives from each group ensured the platform addresses their specific needs while balancing our overall project goals.