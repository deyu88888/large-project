\chapter{Specifications}

This chapter outlines the functional and non-functional specifications of our student society management platform. As follows are all out implementation features

\section{Functional Specifications}

\subsection{User Authentication and Profile Management}
\begin{itemize}
    \item \textbf{University Email Validation}: The system currently checks that email addresses end with the university domain (e.g., @kcl.ac.uk) so only university students can register.
    \item \textbf{Email Verification}: Registration includes an OTP (One-Time Password) step to confirm email ownership.
    \item \textbf{Profile Management}: Users can set their username once during registration, and can set and change their password and academic major later.
    \item \textbf{Student Connections}: Students can search, follow and view other students' profiles.
\end{itemize}

\subsection{Student Dashboard}
\begin{itemize}
    \item \textbf{Activity Summary}: The dashboard shows society memberships, upcoming events and notifications.
    \item \textbf{Society Management}: Students can see societies they've joined and leave them whenever they want, unless they hold a societal role, which in this case they have to resign first.
    \item \textbf{Event Management}: Students can view events and RSVP to attend them.
    \item \textbf{Notifications}: The system groups all society communications and tracks read/unread status.
    \item \textbf{News}: Students can see news from societies they've joined.
    \item \textbf{Society Creation}: Students can propose new societies for admin approval.
    \item \textbf{Search}: Users can search for events and societies.
\end{itemize}

\subsection{Society Management Tools}
\begin{itemize}
    \item \textbf{Society Details}: Society managers can edit the name, description, category, social media links, images, icons and tags (subject to admin approval).
    \item \textbf{Event Management}: Managers can create, edit and manage society events through an approval process.
    \item \textbf{Member Management}: Tools to review membership requests and manage existing members and assign roles to them.
    \item \textbf{News System}: Managers can create, draft, publish and manage news posts.
    \item \textbf{Awards}: System for giving awards to members.
    \item \textbf{Preview}: Managers can preview society pages and events before submitting for approval.
    \item \textbf{Admin Communication}: Direct channel to report issues to university administrators.
\end{itemize}

\subsection{Admin Dashboard}
\begin{itemize}
    \item \textbf{Platform Stats}: Graphs showing active users, events and pending requests.
    \item \textbf{Student Management}: A student database with filtering, search and admin actions.
    \item \textbf{Society Approval}: Tools to review and approve new societies and changes to existing ones.
    \item \textbf{Event Approval}: Interface to review event requests and check they comply with university policies.
    \item \textbf{Report Handling}: System to handle and respond to user reports.
    \item \textbf{Admin Team}: Tools to manage admin permissions and roles.
    \item \textbf{Activity Log}: Tracking of admin actions with undo option.
\end{itemize}

\subsection{Recommendation System}
\begin{itemize}
    \item \textbf{Similarity Scoring}: Combines neural embeddings (35\%), TF-IDF cosine similarity (25\%), keyword overlap (15\%), word overlap (5\%), and domain relationships (20\%).
    \item \textbf{Diversity}: Uses Maximal Marginal Relevance to balance relevance with variety.
    \item \textbf{Activity Weighting}: Gives priority to societies with recent activity.
    \item \textbf{Feedback Processing}: Collects explicit ratings and implicit behaviour signals.
    \item \textbf{Learning}: Recommendations improve based on user interactions.
\end{itemize}

\section{Non-functional Specifications}

\subsection{Usability}
\begin{itemize}
    \item \textbf{Desktop Design}: Platform works best on desktop, with basic mobile support.
    \item \textbf{Display Options}: Users can switch between light and dark modes.
    \item \textbf{Navigation}: Common navigation patterns across the system.
\end{itemize}

\subsection{Security and Privacy}
\begin{itemize}
    \item \textbf{Permission Levels}: Users can only access functions appropriate to their role.
    \item \textbf{Data Protection}: We protect personal information by hiding user profiles from non-logged-in users, concealing email addresses, encrypting tokens, and anonymising sensitive data where needed.
    \item \textbf{Content Approval}:  Approval for society creation, events and content.
    \item \textbf{Action Tracking}: Admin system logs important actions.
\end{itemize}

\subsection{Performance and Reliability}
\begin{itemize}
    \item \textbf{Update Mechanism}: System provides content updates with minimal delay.
    \item \textbf{Multiple Users}: System handles many users at once without slowing down.
    \item \textbf{Error Handling}: Data validation and error handling throughout.
\end{itemize}

\subsection{Scalability}
\begin{itemize}
    \item \textbf{User Numbers}: Architecture supports the entire university student population.
    \item \textbf{Content Growth}: No limits on the number of societies or events.
    \item \textbf{History}: Past events and society information kept with suitable archiving.
\end{itemize}

\subsection{Maintainability}
\begin{itemize}
    \item \textbf{Components}: System built with separate components for easier updates.
    \item \textbf{Technology}: Django and React used consistently across the application.
    \item \textbf{Code Structure}: Designed with cohesion and coupling in mind to aid future development
\end{itemize}

\section{Out of Scope}
While our system covers society management thoroughly, we decided these features were outside our project scope:

\begin{itemize}
    \item \textbf{Financial Tools}: Budget tracking and financial reporting for societies.
    \item \textbf{Ticketing}: Integration with ticket platforms for paid events.
    \item \textbf{Mobile Apps}: Native applications for iOS and Android.
    \item \textbf{Learning Systems}: Connection with university course platforms.
\end{itemize}