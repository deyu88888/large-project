\chapter{Specifications}

This chapter outlines the functional and non-functional specifications of our student society management platform. All features described below have been successfully implemented and deployed in the final system.

\section{Functional Specifications}

\subsection{User Authentication and Profile Management}
\begin{itemize}
    \item \textbf{University-Restricted Registration}: System validates that email addresses end with the university domain (e.g., @kcl.ac.uk) to ensure only legitimate university students can register.
    \item \textbf{Two-Step Verification}: Registration process includes email verification through OTP (One-Time Password) to confirm email ownership.
    \item \textbf{Profile Customization}: Users can set and modify their username, password, and academic major.
    \item \textbf{Social Connections}: Students can search for, follow, and view profiles of other students, facilitating community building.
\end{itemize}

\subsection{Personalized Student Dashboard}
\begin{itemize}
    \item \textbf{Activity Overview}: Dashboard displays membership counts, upcoming events, and unread notifications.
    \item \textbf{Society Management}: Students can view societies they've joined with options to leave if desired.
    \item \textbf{Event Tracking}: Comprehensive display of all events with RSVP functionality and attendance management.
    \item \textbf{Notification System}: Centralizes all society-related communications with read/unread status tracking.
    \item \textbf{News Feed}: Displays news published by societies the student has joined.
    \item \textbf{Calendar Integration}: Visual calendar interface showing all events for better planning.
    \item \textbf{Society Application}: Interface for students to propose new societies, requiring administrative approval.
    \item \textbf{Search Functionality}: Allows searching for events and societies across the platform.
\end{itemize}

\subsection{Society Management Tools}
\begin{itemize}
    \item \textbf{Society Detail Management}: The society managers can edit the name, description, category, social media links, images, icons, and tags of the society (subject to admin approval).
    \item \textbf{Event Creation and Management}: Interface for creating, editing, and managing society events with approval workflows.
    \item \textbf{Membership Administration}: Tools for reviewing pending membership requests and managing existing members.
    \item \textbf{News Publication System}: Functionality to create, draft, publish, and manage society news posts.
    \item \textbf{Member Recognition}: System for assigning awards and roles to society members.
    \item \textbf{Preview Functionality}: Live preview of the society page and events before submission for approval.
    \item \textbf{Administrative Communication}: Direct reporting channel to university administrators for issue resolution.
\end{itemize}

\subsection{Administrative Oversight Dashboard}
\begin{itemize}
    \item \textbf{Platform Analytics}: Real-time metrics showing active users, events, and pending requests.
    \item \textbf{Student Management}: Comprehensive student database with filtering, search, and administrative actions.
    \item \textbf{Society Approval Workflow}: Tools for reviewing and approving new societies and society modification requests.
    \item \textbf{Event Approval System}: Interface for reviewing event requests to ensure compliance with university policies.
    \item \textbf{Report Management}: System for handling and responding to reports from users.
    \item \textbf{Administrative Team Management}: Tools for managing admin permissions and responsibilities.
    \item \textbf{Activity Logging}: Comprehensive tracking of administrative actions with undo functionality.
\end{itemize}

\subsection{AI-Powered Recommendation System}
\begin{itemize}
    \item \textbf{Multi-faceted Similarity Scoring}: Combines neural embeddings (35\%), TF-IDF cosine similarity (25\%), keyword overlap (15\%), basic word overlap (5\%), and domain-specific semantic relationships (20\%).
    \item \textbf{Diversity Algorithm}: Implements Maximal Marginal Relevance to balance relevance with variety in recommendations.
    \item \textbf{Temporal Awareness}: Prioritizes societies with recent activity in recommendations.
    \item \textbf{Adaptive Feedback System}: Collects and processes both explicit ratings and implicit user behavior signals.
    \item \textbf{Personalization}: Recommendations improve over time based on user interactions and preferences.
\end{itemize}

\section{Non-functional Specifications}

\subsection{Usability}
\begin{itemize}
    \item \textbf{Desktop-Optimized Design}: Platform is designed primarily for desktop use, with limited mobile responsiveness. 
    \item \textbf{Accessibility Features}: Interface follows accessibility best practices for inclusive user experience.
    \item \textbf{Appearance Customization}: Users can switch between light and dark modes to enhance readability.
    \item \textbf{Consistent Navigation}: Intuitive navigation patterns maintained across all system interfaces.
\end{itemize}

\subsection{Security and Privacy}
\begin{itemize}
    \item \textbf{Role-Based Access Control}: Multi-tiered permission structure ensuring users only access appropriate functionality.
    \item \textbf{Data Protection}: Personal information is securely stored with multiple protection measures: user profiles are not searchable by non-logged-in users, email addresses are hidden from other users, authentication tokens are encrypted, and sensitive data is appropriately anonymized where necessary.
    \item \textbf{Approval Workflows}: Multi-step approval processes for society creation, event scheduling, and content publication.
    \item \textbf{Activity Monitoring}: Administrative logging of system actions.
\end{itemize}

\subsection{Performance and Reliability}
\begin{itemize}
    \item \textbf{Real-time Updates}: WebSocket integration for immediate notification delivery and content updates.
    \item \textbf{Concurrency Management}: System handles multiple simultaneous users without degradation.
    \item \textbf{Data Integrity}: Robust data validation and error handling throughout the system.
\end{itemize}

\subsection{Scalability}
\begin{itemize}
    \item \textbf{User Capacity}: Architecture designed to accommodate the entire university student population.
    \item \textbf{Society Growth}: No artificial limits on the number of societies or events the system can manage.
    \item \textbf{Data Retention}: Historical events and society information maintained with appropriate archiving.
\end{itemize}

\subsection{Maintainability}
\begin{itemize}
    \item \textbf{Component Modularity}: System designed with clear separation of concerns for easier updates.
    \item \textbf{Technology Stack Standardization}: Consistent use of Django and React frameworks across the application.
    \item \textbf{Code Organization}: Logical structure facilitating future extensions and modifications.
\end{itemize}

\section{Out of Scope}
While the implemented system provides comprehensive society management capabilities, the following features were considered but determined to be outside the project scope:

\begin{itemize}
    \item \textbf{Financial Management}: Tools for society budget tracking and financial reporting.
    \item \textbf{External Event Ticketing}: Integration with third-party ticketing platforms for paid events.
    \item \textbf{Mobile Applications}: Native mobile applications for iOS and Android platforms.
    \item \textbf{Integration with University Learning Management Systems}: Direct connection with academic course platforms.
\end{itemize}

These features represent potential future enhancements that would build upon the current system capabilities while maintaining alignment with the core project objectives.