\chapter{Project Management}
\label{chap:project-management}

This chapter looks at how we managed our university society platform project, from our overall approach to the specific challenges we faced and how we solved them.

\section{Project Management Approach}

\subsection{Our Agile-Inspired Approach}

We used an approach loosely based on Agile principles, but adapted to fit our team's needs and university workloads. We chose this for several reasons:

\begin{itemize}
    \item \textbf{Evolving Requirements}: When we started, we had a clear idea of what the platform should do, but knew many details would change as we better understood what users needed.
    
    \item \textbf{Mixed Experience Levels}: Our team members had varying coding experience. Regular communication helped less experienced members learn from others.
    
    \item \textbf{Different User Roles}: Our platform needed to work for students, society leaders and administrators, which meant we needed a flexible approach to handle these distinct interfaces and permission levels.
\end{itemize}

\subsection{How We Worked}

Our actual working method was quite practical:

\begin{itemize}
    \item \textbf{Regular Meetings}: We met twice a week for 2-3 hours to brainstorm, make decisions and review progress.
    
    \item \textbf{Meeting Notes}: We kept detailed notes of all meetings to track decisions and tasks.
    
    \item \textbf{Group Chat}: Daily WhatsApp messages helped us solve urgent problems and make decisions between meetings.
    
    \item \textbf{Development Phases}: Our work naturally fell into three stages: building core features (first 5 weeks), refining and completing features (next 3 weeks), and fixing bugs (final 3 weeks).
\end{itemize}

\subsection{What Worked and What Didn't}

Our approach had strengths and weaknesses:

\begin{itemize}
    \item \textbf{Flexibility}: Not having rigid processes meant we could quickly shift focus when needed, especially when moving from building features to fixing bugs.
    
    \item \textbf{Communication Issues}: While our group chat was useful, important information sometimes got buried. Our twice-weekly meetings helped get everyone back on the same page.
    
    \item \textbf{Value of Notes}: Our meeting notes proved very helpful, especially when team members missed meetings due to other commitments.
\end{itemize}

\section{Task Management}

\subsection{Tools We Used}

We kept our toolset simple:

\begin{itemize}
    \item \textbf{Discord}: For quick, unplanned meetings and discussions
    
    \item \textbf{Microsoft Teams}: For scheduled online meetings
    
    \item \textbf{WhatsApp}: For daily messages and quick decisions
    
    \item \textbf{GitHub}: For code review and version control

    \item \textbf{Postman}: For checking endpoints
\end{itemize}

Tasks were organised naturally rather than with formal methods. We maintained a list of features to build, and team members generally picked tasks that matched their skills.

\subsection{Development Assistance Tools}

To help manage our development process efficiently within our time constraints, we employed several productivity tools:

\begin{itemize}
    \item \textbf{AI Assistance}: We used AI tools like ChatGPT for troubleshooting complex bugs, getting second opinions on code structure, and helping with debugging sessions when we were stuck on particularly difficult problems. This proved valuable when we couldn't afford to spend excessive time on issues that were blocking progress.
    
    \item \textbf{Code Formatting Tools}: We used tools like autopep8 to maintain consistent code formatting and catch potential issues early.
    
\end{itemize}

Our approach to using AI was practical—we used it as a supplementary resource rather than a replacement for our own problem-solving. We found it particularly helpful for:

\begin{itemize}
    \item Overcoming complex issues when debugging 
    \item Validating our architectural decisions
    \item Providing quick feedback on code quality
\end{itemize}

This balanced approach allowed us to keep our original code intact while leveraging available tools to overcome obstacles efficiently.

\subsection{Lessons from Task Management}

We learned several lessons about task management:

\begin{itemize}
    \item \textbf{Over-ambitious Planning}: In the first few weeks, we planned more features than we could actually build, which caused some stress and disappointment.
    
    \item \textbf{Better Task Breakdown}: We started with vague tasks like "Build the student dashboard," but learned to break these into smaller pieces, which made tracking progress easier.
    
    \item \textbf{Quality Standards}: We gradually developed better standards for when a feature was "done," including:
    \begin{itemize}
        \item Code review by another team member
        \item Rigorous testing
        \item Updated documentation
    \end{itemize}
\end{itemize}

\section{Team Organisation}

\subsection{How We Organised Ourselves}

We kept our team structure flexible:

\begin{itemize}
    \item \textbf{Natural Leaders}: Instead of formal roles, we let leadership emerge based on who knew most about different aspects of the project.
    
    \item \textbf{Playing to Strengths}: Team members worked on parts that matched their skills (frontend, backend, etc.).
    
    \item \textbf{Group Decisions}: Big decisions were made together during our meetings, aiming for consensus.
\end{itemize}

This approach let us adapt as the project evolved while using everyone's skills effectively.

\subsection{Team Challenges}

We faced several challenges in how we worked together:

\begin{itemize}
    \item \textbf{Knowledge Hoarding}: By mid-project, we realised some critical parts were understood by only one person. We tried to fix this through better documentation and explanations, though it remained an issue.
    
    \item \textbf{Uneven Workloads}: Letting people choose their tasks sometimes led to unbalanced work, with some team members taking on much more than others.
    
    \item \textbf{Asynchronous Development}: Different schedules meant much work happened independently. We improved this by:
    \begin{itemize}
        \item Writing clearer commit messages
        \item Sharing summaries of new code in the group chat
        \item Posting screenshots of new features
    \end{itemize}
\end{itemize}

\section{Risk Management}

\subsection{How We Handled Risks}

We took an informal approach to risks, discussing them as they came up rather than through formal processes:

\begin{itemize}
    \item \textbf{User Role Management}: We knew that building different interfaces for students, society leaders and admins would be tricky. We addressed this by deciding early on how permissions would work and what each role could do.
    
    \item \textbf{Feature Selection}: With many possible features to build, we had to decide what to prioritise. We tried to focus on core functions that would give users the most value.
    
\end{itemize}

\subsection{Risk Management Results}

Our approach to risks had mixed outcomes:

\begin{itemize}
    \item \textbf{Feature Priorities}: Early on, we sometimes chose features based on what was technically interesting rather than what users needed most. We got better at focusing on core user needs as the project went on.
    
    \item \textbf{Code Quality Issues}: In our rush to add features, we sometimes wrote code that needed refactoring later. Our final two weeks focused heavily on fixing and improving existing code.
    
    \item \textbf{Role System Success}: Our early work on user roles and permissions paid off. By planning this carefully from the start, we avoided major rework later and built consistent interfaces for each user type.
\end{itemize}

\section{Progress Tracking}

\subsection{How We Tracked Progress}

We kept progress tracking simple:

\begin{itemize}
    \item \textbf{Trello Board Tracking}: We maintained a Trello board with cards for each feature, moving them through columns for 'To Do', 'In Progress', and 'Done' to visualize our progress
    \item \textbf{GitHub Activity}: Looking at commits and pull requests to see development progress
    
    \item \textbf{Demos}: Showing working features during our meetings
\end{itemize}

This approach matched our overall flexible management style while giving us a good general view.

\subsection{Progress Insights}

Our approach to tracking revealed some patterns:

\begin{itemize}
    \item \textbf{Features vs Quality}: Our focus on completing features sometimes meant code quality suffered. Later in the project, we shifted to improving existing features rather than adding new ones.
    
    \item \textbf{Work Patterns}: We noticed that most work happened just before our meetings, as everyone prepared to show their progress. This created an uneven pace but provided natural deadlines.
\end{itemize}

\section{Key Lessons for Future Projects}

Based on our experience, we'd make these changes for future projects:

\begin{itemize}
    \item \textbf{More Quality Control}: While flexibility worked well, having more structure around code reviews and quality checks would have saved us from spending time fixing things later.
    
    \item \textbf{Better Knowledge Sharing}: We should have established more formal ways to share knowledge early on, like documentation standards or pair programming.
    
    \item \textbf{Planned Refactoring Time}: We should have scheduled time for cleaning up code throughout the project rather than leaving it all to the end.
\end{itemize}

Overall, our flexible approach worked well for our project constraints. By maintaining regular communication and being open to adapting our process along the way, we managed to overcome challenges and deliver a working platform.