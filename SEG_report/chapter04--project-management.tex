\chapter{Project Management}
\label{chap:project-management}

This chapter presents, reflects on, and evaluates our team's approach to project management throughout the development of the university society management platform. We discuss our methodology, coordination mechanisms, challenges encountered, and critical reflections on the effectiveness of our approaches.

\section{Project Management Approach}

\subsection{Flexible Agile-Inspired Methodology}

Our team adopted a flexible Agile-inspired approach to project management, drawing on iterative principles while adapting them to fit our team's specific needs and academic constraints. This decision was based on several factors:

\begin{itemize}
    \item \textbf{Uncertain Requirements}: At project initiation, we recognized that while we had a clear vision of the platform's purpose, many specific requirements would evolve as we better understood stakeholder needs.
    
    \item \textbf{Team Composition}: Our team consisted of members with varied experience levels in software development. Our emphasis on frequent communication and knowledge sharing provided a structure for less experienced members to learn from those with more expertise.
    
    \item \textbf{Project Complexity}: The multi-tiered nature of our platform (serving different user roles with distinct needs) suggested that a sequential approach would be less effective than an iterative one that allowed us to refine features based on feedback.
\end{itemize}

\subsection{Practical Implementation}

While our approach was influenced by Agile principles, our actual implementation was adapted to better fit our team's working style and academic schedules:

\begin{itemize}
    \item \textbf{Regular Collaboration Sessions}: We held twice-weekly team meetings (2-3 hours each) for collaborative brainstorming, decision-making, and progress reviews.
    
    \item \textbf{Documentation}: We maintained comprehensive meeting minutes to document all decisions and action items, ensuring team alignment.
    
    \item \textbf{Continuous Communication}: Daily communication in a group chat addressed immediate problems and facilitated necessary decisions between formal meetings.
    
    \item \textbf{Phased Development Approach}: Our project naturally evolved through three distinct phases: initial development with rapid feature addition (first 5 weeks), feature refinement and completion (middle period), and final refactoring and debugging (final two weeks).
\end{itemize}

\subsection{Critical Reflection on Our Approach}

Our flexible approach had both strengths and limitations:

\begin{itemize}
    \item \textbf{Adaptability Advantage}: The absence of rigid processes allowed us to shift focus quickly when needed, particularly when transitioning from feature development to refinement and debugging.
    
    \item \textbf{Communication Challenges}: While our group chat provided continuous communication channels, important information sometimes got lost in the conversation flow. Our twice-weekly meetings helped realign everyone but occasionally left gaps in coordination between sessions.
    
    \item \textbf{Documentation Value}: Our meeting minutes proved invaluable for maintaining project continuity, especially when team members missed meetings due to personal issues.
\end{itemize}

\section{Task Management and Workflow}

\subsection{Tool Selection and Task Organization}

To manage our project tasks and workflow, we used a combination of lightweight tools:

\begin{itemize}
    \item \textbf{Discord}: For quick unscheduled meetings
    
    \item \textbf{Teams}: For scheduled online meetings
    
    \item \textbf{WhatsApp Group}: For daily communication and quick decision-making
    
    \item \textbf{GitHub Pull Requests}: For code review and quality control
\end{itemize}

Our task organization evolved organically rather than following formal methods. We maintained a prioritized feature list that guided our development focus, with tasks generally self-assigned based on individual interests and expertise.

\subsection{Critical Reflection on Task Management}

Our approach to task management revealed several insights:

\begin{itemize}
    \item \textbf{Initial Overestimation}: During the first few weeks, we consistently overestimated what could be accomplished, committing to more features than we could complete. This led to some feature slippage and occasional team frustration.
    
    \item \textbf{Task Granularity Issues}: Early task definitions were often too broad (e.g., "Implement student dashboard"), making progress difficult to track. We gradually improved at breaking down complex features into more manageable components, which improved both estimation accuracy and work distribution.
    
    \item \textbf{Quality Control Evolution}: Initially, we lacked a clear process for determining when a feature was truly complete. As the project progressed, we developed more rigorous quality standards, including:
    \begin{itemize}
        \item Code review by at least one team member
        \item Manual testing on desktop browsers
        \item Documentation updates
    \end{itemize}
    This improved overall code quality but was implemented reactively rather than from the outset.
\end{itemize}

\section{Team Organization and Collaboration}

\subsection{Team Structure and Work Distribution}

We adopted a flexible team structure based primarily on technical strengths:

\begin{itemize}
    \item \textbf{Natural Leadership}: Rather than assigning formal project management roles, we allowed leadership to emerge naturally based on expertise in different areas
    
    \item \textbf{Technical Specializations}: Team members gravitated toward components matching their strengths (frontend, backend, AI, etc.)
    
    \item \textbf{Collaborative Decision-Making}: Major decisions were made collectively during our twice-weekly meetings, with consensus-building as our primary approach
\end{itemize}

This flexible structure allowed us to adapt to the evolving needs of the project while leveraging each team member's unique skills.

\subsection{Critical Reflection on Team Organization}

Our team organization faced several challenges throughout the project:

\begin{itemize}
    \item \textbf{Knowledge Silos}: By mid-project, we recognized dangerous knowledge silos forming, with only one team member understanding certain critical components. We attempted to address this through more detailed documentation and explanations during meetings, though this remained a challenge throughout the project.
    
    \item \textbf{Workload Distribution}: Our self-selection approach to tasks occasionally led to uneven workload distribution. Some team members took on significantly more complex components than others, creating bottlenecks when those components needed integration.
    
    \item \textbf{Remote Collaboration Challenges}: The team's varied schedules meant that much work was done asynchronously. We initially struggled with communication that would enable effective asynchronous work. After identifying this issue, we improved our practices by:
    \begin{itemize}
        \item Writing more descriptive commit messages
        \item Sharing a summary of the implementation in the group chat
        \item Sharing explanatory screenshots in the group chat when implementing new features
    \end{itemize}
\end{itemize}

\section{Risk Management}

\subsection{Informal Risk Identification and Mitigation}

Rather than conducting formal risk management exercises, our approach to risks was more organic and reactive. We identified several key risks through group discussions:

\begin{itemize}
    \item \textbf{Technical Complexity of Multi-Role System}: We recognized that implementing different user roles (student, society president, admin) with distinct interfaces would be challenging. We addressed this by discussing the permission structure early and establishing clear boundaries between role capabilities.
    
    \item \textbf{Feature Prioritization Challenges}: With numerous possible features to implement, we faced difficult decisions about which to prioritize. We mitigated this by focusing first on core functionality that would provide the most value to users before adding more specialized features.
    
    \item \textbf{Academic Time Constraints}: University coursework deadlines created variable availability throughout the project. We acknowledged this reality and maintained flexibility in our scheduling.
\end{itemize}

\subsection{Critical Reflection on Risk Management}

Our informal approach to risk management had mixed results:

\begin{itemize}
    \item \textbf{Feature Prioritization Challenges}: Our initial approach to feature prioritization was sometimes based more on technical interest than user value. As the project progressed, we improved our prioritization by focusing on core user journeys first. However, this initial lack of structured prioritization meant we spent time on some features that were ultimately less important than others we had to rush later.
    
    \item \textbf{Technical Debt Accumulation}: In our rush to implement features, we accumulated technical debt, particularly in the early phase. This necessity to refactor code later was anticipated but not formally planned for. Our final two weeks became heavily focused on refactoring and debugging as a result.
    
    \item \textbf{Successful Role System Implementation}: Our early attention to defining user role permissions proved beneficial. By discussing the permission structure early and clearly defining role boundaries, we avoided major redesigns later in the project. This allowed us to build different dashboard interfaces for each role with consistent and appropriate access controls.
\end{itemize}

\section{Project Progress Monitoring}

\subsection{Progress Tracking Approach}

Our project progress monitoring relied primarily on:

\begin{itemize}
    \item \textbf{Feature Completion Tracking}: Regular reviews of which features were complete, in progress, or not yet started
    
    \item \textbf{GitHub Activity}: Monitoring commit activity and pull requests to gauge development progress
    
    \item \textbf{Biweekly Demonstrations}: Demonstrating working features during our meetings to verify progress
\end{itemize}

This lightweight approach to progress tracking aligned with our flexible project management methodology while providing sufficient visibility into project status.

\subsection{Critical Reflection on Progress Monitoring}

Our monitoring approach revealed several insights:

\begin{itemize}
    \item \textbf{Feature vs. Quality Tension}: Our focus on feature completion sometimes came at the expense of code quality. In later stages, we adjusted by placing greater emphasis on testing and refinement rather than new feature development.
    
    \item \textbf{Development Rhythm}: We observed that productivity typically increased just before our biweekly meetings, as team members worked to complete features for demonstration. While this created a somewhat uneven development pace, it established natural checkpoints that helped maintain momentum.
   
\end{itemize}

\section{Key Lessons and Future Recommendations}

Based on our project management experience, we identified several key lessons that would inform our approach to future projects:

\begin{itemize}
    \item \textbf{Balanced Structure}: While our flexible approach enabled adaptability, a slightly more structured process—particularly for code reviews and quality assurance—would have reduced technical debt without sacrificing creativity.
    
    \item \textbf{Knowledge Sharing}: More formalized knowledge transfer should be established earlier. Techniques such as pair programming and documentation requirements should be core practices rather than remedial measures.
    
    \item \textbf{Technical Debt Management}: Explicitly allocating time for refactoring and technical debt reduction should be planned from the outset rather than becoming a necessity at the end of the project.
    
\end{itemize}

Overall, our flexible approach provided the adaptability needed to accommodate our team's academic constraints while still delivering a functional platform. Our willingness to communicate frequently and adjust our processes as needed helped us overcome the challenges inherent in a complex project with a diverse team.