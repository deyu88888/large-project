\chapter{Introduction}

\noindent Our project delivers a comprehensive platform that transforms how university students discover, join and manage societies. Universities adopting this system will see tangible benefits: higher levels of student engagement in extracurricular activities, stronger campus communities, and more efficient oversight of student organisations. From the students' perspective, our platform removes traditional obstacles to society participation by offering tailored recommendations and a straightforward interface for interacting with university societies—ultimately enhancing their university experience and creating a stronger sense of belonging on campus.

\noindent As a group, we have constructed a web app that makes use of Django, a python library, for the backend, and ReactJS for the frontend. We implemented various UI features such as a dark/light mode toggle for ease on the eyes, and a sophisticated recommendations system to be sure users quickly find societies that interest them. This is done utilising neural embeddings and domain-specific semantic relations to effectively adapt to the student body's varied interests. Our implementation includes robust user authentication with email verification and a multi-tiered permission structure that serves the distinct needs of students, society leaders and university administrators. The entire solution is deployed through an automated GitHub workflow to a Virtual Private Server, utilising Traefik as a reverse proxy to manage routing and SSL certificates, ensuring both security and reliability.