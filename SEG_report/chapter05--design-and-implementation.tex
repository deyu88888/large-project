\chapter{Design and Implementation}
\label{chap:design-and-implementation}

This chapter covers the main design and implementation choices we made while building our university society platform. We'll look at our system architecture, key design decisions, and how we tackled various technical challenges.

\section{Architecture}

\subsection{Overall System Architecture}

We built our system using a client-server architecture that keeps the frontend and backend separate. This follows the Model-View-Controller (MVC) pattern, which makes the system more modular and easier to maintain. Our architecture has three main parts:

\begin{itemize}
    \item \textbf{Frontend}: A dynamic React application that handles the user interface
    \item \textbf{Backend}: A Django REST API that provides data and handles business logic
    \item \textbf{Database}: SQLite storage for all our application data
\end{itemize}

We chose this setup for several practical reasons:

\begin{itemize}
    \item It allowed team members to work on different parts at the same time
    \item Changes to one part had minimal impact on others
    \item It followed current industry practices for web applications
\end{itemize}

\subsection{Backend Architecture}

Our backend uses Django's project structure with some important customisations:

\begin{itemize}
    \item \textbf{API-First Approach}: All functions are available through a REST API
    \item \textbf{Model-Driven Development}: Database structure defined through Django's ORM
    \item \textbf{Domain-Focused Organisation}: Functions organised by what they do rather than how they do it
    \item \textbf{Separated Files}: Instead of large files, we split models, views, serializers, and tests into separate directories
\end{itemize}

This organisation is clear in our file structure:

\begin{itemize}
    \item \texttt{models\_files/}, \texttt{views\_files/}, and \texttt{serializers\_files/}: Each directory contains files organized by feature or entity type (e.g., \texttt{user\_models.py} for user-related models, \texttt{society\_views.py} for society-related views), keeping related code together for easier maintenance
    \item \texttt{tests/}: Contains organized test files matching the application structure.
\end{itemize}


We used several key design patterns in the backend:

\begin{itemize}
    \item \textbf{Utility Abstraction}: We created utility files for models, views, and serializers to extract common operations like permission checks and error responses, reducing code duplication
    \item \textbf{Signals Architecture}: Django signals used for event-driven functionality
\end{itemize}

\subsection{Frontend Architecture}

Our frontend uses React with a component-based design following these principles:

\begin{itemize}
    \item \textbf{Component Hierarchy}: UI built from reusable components with clear relationships
    \item \textbf{State Management}: Application state handled through React hooks and context
    \item \textbf{Route-Based Organisation}: Features organised around application routes
    \item \textbf{Responsive Design}: UI components that work across different screen sizes
\end{itemize}

\subsection{Deployment Architecture}

We set up an automated deployment system to host our application on a Virtual Private Server (VPS):

\begin{itemize}
    \item \textbf{GitHub Integration}: Automated deployment triggered by GitHub pushes
    \item \textbf{Traefik Reverse Proxy}: Handles routing, SSL certificates, and HTTPS connections
    \item \textbf{Containerised Services}: Frontend and backend run as separate services
    \item \textbf{SQLite Database}: Stores all persistent application data
\end{itemize}

\section{Key Design Decisions}

\subsection{Multi-Tiered Dashboard Design}

One of our most important design decisions was creating different dashboards for different user roles. This required:

\begin{itemize}
    \item \textbf{Role-Based Access}: Each dashboard only shows functions that role can use
    \item \textbf{Shared Components}: Common parts reused across dashboards
    \item \textbf{Consistent Look}: Similar design across all dashboards despite different functions
\end{itemize}

We took an approach that was a balance between:

\begin{enumerate}
    \item One dashboard with conditional elements based on role
    \item Completely separate dashboards with no shared code
\end{enumerate}

This in practice meant separate dashboards for different roles, but with shared components where possible. This reduced duplicate code while keeping the role-specific functions clearly separated.

\subsection{Data Model Design}

Our data model focused on the relationships between users, societies, events, and admin functions:

\begin{itemize}
    \item \textbf{Flexible Roles}: A single user can have different and simultaneous relationships with societies (member, society manager)
    \item \textbf{Approval Workflows}: Built-in states for societies, events, and news that need admin approval
    \item \textbf{Date and Time Handling}: Proper datetime management for events and calendar features
\end{itemize}

This design supports the complex relationships in university societies while keeping data consistent and securing access.

\subsection{Notification System}

To keep users informed, we implemented a notification system:

\begin{itemize}
    \item \textbf{User Alerts}: Notifications for society updates, event changes, and membership status
    \item \textbf{Notification Centre}: Central location for users to view all their notifications
    \item \textbf{Read/Unread Status}: Tracking which notifications have been seen
\end{itemize}

This system helps users stay updated on society activities without needing to constantly check for changes.

\section{Implementation Strategies}

\subsection{Authentication and Authorisation}

Implementing authentication and authorisation was critical for our multi-role system:

\begin{itemize}
    \item \textbf{Token-Based Authentication}: JWT tokens for secure login
    \item \textbf{University Email Verification}: OTP verification ensures only university students can register
    \item \textbf{Detailed Permissions}: Specific permissions for different actions rather than broad role access
\end{itemize}

This approach secures our system while keeping the user experience smooth. The token system works well with our separate frontend and backend.

\subsection{API Design and Frontend Integration}

Our API design focused on creating intuitive endpoints that matched frontend needs:

\begin{itemize}
    \item \textbf{Data Transformation}: Serializers to convert between Django models and JSON
    \item \textbf{Nested Resources}: Endpoints structured to reduce the need for multiple requests
\end{itemize}

This design made frontend development easier by providing well-structured data that matched what components needed.

\section{Technical Challenges and Solutions}

\subsection{Code Organisation for Maintainability}

As our code grew, keeping it organised became important:

\begin{itemize}
    \item \textbf{Challenge}: Keeping related classes together while preventing files from becoming too large
    \item \textbf{Solution}: A modular file structure with directories grouped by function
\end{itemize}

Our approach of separating views, serializers, and models into dedicated files and directories made code easier to find and maintain.

\subsection{Testing and Quality Assurance}

Ensuring system quality required thorough testing:

\begin{itemize}
    \item \textbf{Challenge}: Testing complex interactions between components and user roles
    \item \textbf{Solution}: Layered testing strategy with unit tests, Postman and manual testing
\end{itemize}

Our tests are organised in the same way as our model, serializer and views directories making tests easier to find.

\subsection{Code Structure and Function Length}
While we aimed for short, focused functions, some view methods were naturally longer due to the complex roles they handle:

\begin{itemize}

    \item \textbf{Permission Handling}: Functions dealing with complex permission checking and role assignments needed comprehensive logic in one place
    
    \item \textbf{AI Recommendation System}: Our recommendation-related files contain longer functions due to the complexity of the AI implementation, which was outside our usual scope of expertise. With limited time to explore these techniques in depth, we prioritised functionality over code structure refinement
\end{itemize}

For example, the membership approval function in \texttt{president\_views.py}, where the \texttt{Pending
MembersView.post} method handles permission verification, request validation, society membership updates, and appropriate response generation – all functions that should be grouped together as they belong exclusively to this view.

Similarly, recommendation-related functions like those in \texttt{recommendation\_views.py} manage complex data processing for generating personalised society suggestions, tracking user interactions, and processing feedback to improve future recommendations.

\section{Evolution of Implementation}

Our implementation evolved throughout the project:

\begin{itemize}
    \item \textbf{Initial Phase}: Built core models, basic operations, and role-specific dashboards
    \item \textbf{Middle Phase}: Refined existing features and added complex business logic and approval workflows
    \item \textbf{Final Phase}: Polished the user experience, optimised performance, and fixed bugs
\end{itemize}

This approach let us deliver a working system early while adding more sophisticated features over time.

\section{Reflections on Design and Implementation}

There were several key decisions that shaped our project:

\begin{itemize}
    \item \textbf{Positive Impact}: Separating frontend and backend allowed parallel development and clear boundaries
    \item \textbf{Challenge}: Some features added complexity but significantly improved the user experience
    \item \textbf{Learning}: Earlier focus on file organisation would have reduced later refactoring work
\end{itemize}

These insights will help us approach future projects more effectively, particularly in making architectural decisions that support the entire project.